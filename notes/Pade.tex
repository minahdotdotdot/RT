\documentclass{article}
%Packages
\usepackage[utf8]{inputenc}
\usepackage{geometry, graphicx,wrapfig}
\usepackage{enumerate}
\usepackage{amsmath,amssymb,amsfonts,amsthm, bm}
\usepackage{xcolor} %just for visible comments.
\usepackage[linesnumbered,ruled,vlined]{algorithm2e}
\usepackage[toc,page]{appendix}
\usepackage{makecell}

% New theorems and commands
\newtheorem{theorem}{Theorem}[section]
\newtheorem{lemma}[theorem]{Lemma}
\newtheorem{corollary}[theorem]{Corollary}
\newtheorem{assump}[theorem]{Assumption}
\newcommand\mycommfont[1]{\ttfamily\textcolor{orange}{#1}}
\newcommand{\R}{\mathbb{R}}
\newcommand{\F}{\mathbb{F}}
\newcommand{\dd}{\delta}
\newcommand{\tth}{\theta}
\newcommand{\bb}[1]{\mathbf{#1}}
\newcommand{\fl}{\mathrm{fl}}
\newcommand{\ddt}{\frac{\mathrm{d}}{\mathrm{d}t}}
\SetCommentSty{mycommfont}
\theoremstyle{definition}
\newtheorem{definition}{Definition}[section]
\newcommand\comment[1]{\color{blue}#1}
% Document
\title{Pad\`{e} Approximation of Taylor Expansion of functions}
\author{asdf}
%\date{\today}

\begin{document}
\maketitle
These notes show in detail Example 1 of Bengt Fornberg's notes on Pad\'{e} approximations of polynomials.
A Pad\'{e} approximation of order (M,N),
\begin{equation}
P_M^N(x) = \frac{\sum_{n=0}^N a_nx^n}{1+\sum_{m=1}^M b_nx^n},
\end{equation}
generalizes a Taylor expansion of degree $M+N$.

Suppose that we have available to us an $M+N$ degree Taylor expansion of a function, 
\begin{equation}
T_{N+M}(x) = \sum_{j=0}^{N+M}c_jx^n,
\end{equation}
but want to approximate it with a rational function where the numerator has degree $N$ and denominator has degree $M$.

We set $T_{N+M}(x)$ and $P_M^N(x)$ equal, but multiply both sides by $1+\sum_{m=1}^M b_nx^n$.
\begin{align}
\sum_{n=0}^N a_nx^n = \left(1+\sum_{m=1}^M b_nx^n\right) \sum_{j=0}^{N+M}c_jx^n
\end{align}
\section{Case 1: $N\leq M$}
We match the LHS coefficients to the coefficients of the product of sums.
\begin{align*}
a_0 &= c_0\\
a_1 &= c_1 + b_1c_0\\
\vdots &= \quad\vdots\\
a_N &=c_N + b_1c_{N-1} + \cdots + b_Nc_0
\end{align*}

Since the numerator is only an $N$-degree polynomial, the remaining  $M$ equations are formed as,
\begin{align*}
0 &= c_{N+1} + b_1 c_{N} + \cdots + b_{N+1} c_0\\
\vdots &= \quad\vdots\\
0 &= c_{M} + b_1 c_{M-1} + \cdots + b_{M}c_0\\
0 &= c_{M+1} + b_1 c_{M} + \cdots + b_{M}c_1\\
\vdots &= \quad\vdots\\
0 &= c_{M+N} + b_1 c_{M+N-1} + \cdots + b_{M}c_N
\end{align*}

This system of $M$ equations can be formulated as an inverse problem, since all the $c$'s are known. 
\[\begin{bmatrix}
c_{N} & c_{N-1} & \cdots & c_{0} & 0 & \cdots & 0\\
c_{N+1} & c_{N} & \cdots & c_{1} & c_0 & \cdots & 0\\
\vdots & \vdots & \vdots & \vdots & \vdots & \ddots & \vdots\\
c_{M-1} & c_{M-2} & \cdots & c_{M-N-1} & c_{M-N-2} & \cdots & c_0 \\
c_{M} & c_{M-1} & \cdots & c_{M-N} & c_{M-N-1} & \cdots & c_1 \\
\vdots & \vdots & \vdots & \vdots & \vdots & \vdots & \vdots\\
c_{M+N-1} & c_{M+N-2} & \cdots & c_{M} & c_{M-1} & \cdots & c_{N-1}
\end{bmatrix}\begin{bmatrix}
b_1\\
b_2\\
\vdots \\
b_{N+1}\\
b_{N+2}\\
\vdots\\
b_M
\end{bmatrix} = \begin{bmatrix}
-c_{N+1}\\
-c_{N+2}\\
\vdots \\
-c_{M}\\
-c_{M+1}\\
\vdots \\
-c_{M+N}
\end{bmatrix}\]

Once we have solved for $b$'s, we can easily solve for all the $a$ coefficients. 

\section{Case 1: $N> M$}
We match the LHS coefficients to the coefficients of the product of sums.
\begin{align*}
a_0 &= c_0\\
a_1 &= c_1 + b_1c_0\\
\vdots &= \quad\vdots\\
a_M &=c_M + b_1c_{M-1} + \cdots + b_Mc_0 \\
a_{M+1}&= c_{M+1} + b_1 c_M + \cdots b_Mc_1 \\
a_{N}&= c_N + b_1c_{N-1} + \cdots b_M c_{N-M}
\end{align*}

Since the numerator is only an $N$-degree polynomial, the remaining $M$ equations are formed as,
\begin{align*}
0 &= c_{N+1} + b_1 c_{N} + \cdots + b_{M} c_{N+1-M}\\
0 &= c_{N+2} + b_1 c_{N+1}+ \cdots + b_{M} c_{N+2-M}\\
\vdots &= \quad\vdots\\
0 &= c_{M} + b_1 c_{M-1} + \cdots + b_{M}c_0\\
0 &= c_{M+1} + b_1 c_{M} + \cdots + b_{M}c_1\\
\vdots &= \quad\vdots\\
0 &= c_{M+N} + b_1 c_{M+N-1} + \cdots + b_{M}c_N
\end{align*}

This system of $M$ equations can be formulated as an inverse problem, since all the $c$'s are known. 
\[\begin{bmatrix}
c_{N} & c_{N-1} & \cdots & c_{N-(M-1)}\\
c_{N+1} & c_{N} & \cdots & c_{N-(M-2)}\\
\vdots & \vdots & \vdots & \vdots \\
c_{N+M-1} & c_{M+N-2} & \cdots & c_N
\end{bmatrix}\begin{bmatrix}
b_1\\
b_2\\
\vdots \\
b_M
\end{bmatrix} = \begin{bmatrix}
-c_{N+1}\\
-c_{N+2}\\
\vdots \\
-c_{M+N}
\end{bmatrix}\]

Once we have solved for $b$'s, we can easily solve for all the $a$ coefficients. 

\end{document}