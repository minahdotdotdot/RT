\documentclass{article}
%Packages
\usepackage[utf8]{inputenc}
\usepackage{geometry, graphicx,wrapfig}
\usepackage{enumerate}
\usepackage{amsmath,amssymb,amsfonts,amsthm, bm}
\usepackage{xcolor} %just for visible comments.
\usepackage[linesnumbered,ruled,vlined]{algorithm2e}
\usepackage[toc,page]{appendix}
\usepackage{makecell}

% New theorems and commands
\newtheorem{theorem}{Theorem}[section]
\newtheorem{lemma}[theorem]{Lemma}
\newtheorem{corollary}[theorem]{Corollary}
\newtheorem{assump}[theorem]{Assumption}
\newcommand\mycommfont[1]{\ttfamily\textcolor{orange}{#1}}
\newcommand{\R}{\mathbb{R}}
\newcommand{\F}{\mathbb{F}}
\newcommand{\dd}{\delta}
\newcommand{\tth}{\theta}
\newcommand{\bb}[1]{\mathbf{#1}}
\newcommand{\fl}{\mathrm{fl}}
\newcommand{\ddt}{\frac{\mathrm{d}}{\mathrm{d}t}}
\SetCommentSty{mycommfont}
\theoremstyle{definition}
\newtheorem{definition}{Definition}[section]
\newcommand\comment[1]{\color{blue}#1}
% Document
\title{Integrating Factor Methods}
%\date{\today}

\begin{document}
\maketitle
We study ODEs of type
\begin{equation}
	\ddt y = Ly + N(y),
\end{equation}
where $y$ may be a vector, $L$ is a linear operator, and $N$ is a nonlinear operator. 
\section{IF Methods}
One form of the Integrating Factor method makes the change of variables $v(t)=e^{(t_{n-1}-t)L}y(t)$, which results in a new form.
\begin{equation}
	\ddt v = f(v,t) = e^{(t_{n-1}-t)L} N(e^{(t-t_{n-1})L} v)
	\label{eqn:tnIF}
\end{equation}

Note that while $t_{n-1}$ can be replaced by any constant, this choice helps simplify the discretization of Equation~\ref{eqn:tnIF}.
For example, at $t=t_{n-1}$, we have exactly $y(t_{n-1})=v(t_{n-1})$. 
We will simplify the above notation as $y(t_n) = y_n$ and $v(t_n) = v_n$. 


This class of methods apply explicit methods to Equation \ref{eqn:tnIF}, then returns to the original variables. 

\paragraph{IF-Euler} Applying explicit Euler to Equation~\ref{eqn:tnIF} results in,
\begin{equation*}
v_n = v_{n-1} + h f(v_{n-1, t_{n-1}}) = v_{n-1} + h N(v_{n-1}). 
\end{equation*}

Returning to the original variables gives us the full IF-Euler discretization.
\begin{equation}
y_n = e^{hL}v_n =e^{hL} (y_{n-1}+hN(y_{n-1}))
\end{equation}

\paragraph{IF-RK2 (Heun/Trapezoidal)} This is one of the simplest non-trivial Runge-Kutta method. 
\begin{align*}
k_1 &= f(v_{n-1}, t_{n-1})\\
k_2 &= f(v_{n-1}+hk_1, t_{n-1}+h)\\
v_n &= v_{n-1} + h\left(\frac{1}{2}k_1 +\frac{1}{2}k_2\right)
\end{align*}

Converting back to original variables:
\begin{align*}
k_1 &= f(v_{n-1}, t_{n-1})=N(y_{n-1})\\
k_2 &= f(v_{n-1}+hk_1, t_{n-1}+h) =e^{-hL}N(e^{hL}(y_{n-1}+hk_1))\\
y_n &= e^{hL}v_n  = e^{hL}\left(y_{n-1} + h\left(\frac{1}{2}k_1 +\frac{1}{2}k_2\right)\right)\\
&=  e^{hL} \left(y_{n-1} +\frac{h}{2}N(y_{n-1})\right)+\frac{h}{2}N(e^{hL}(y_{n-1}+hN(y_{n-1})))
\end{align*}
%\paragraph{IF-RK2 (Explicit Midpoint Method)} This is one of the simplest non-trivial Runge-Kutta method. 
%\begin{align*}
%k_1 &= f(v_{n-1}, t_{n-1})\\
%k_2 &= f(v_{n-1}+\frac{h}{2}k_1, t_{n-1}+\frac{h}{2})\\
%v_n &= v_{n-1} + hk_2
%\end{align*}
%
%Converting back to original variables:
%\begin{align*}
%k_1 &= f(v_{n-1}, t_{n-1}) = N(y_{n-1})\\
%k_2 &= f(v_{n-1}+\frac{h}{2}k_1, t_{n-1}+\frac{h}{2})\\
%&= e^{-\frac{1}{2}hL}N(e^{\frac{1}{2}hL}(y_{n-1}+\frac{h}{2}k_1))\\
%y_n &= e^{hL}v_n = e^{hL}(y_{n-1} + hk_2)
%\end{align*}
%
%\paragraph{IF-RK2 (Ralston's Method)} This is one of the simplest non-trivial Runge-Kutta method. 
%\begin{align*}
%k_1 &= f(v_{n-1}, t_{n-1})\\
%k_2 &= f(v_{n-1}+\frac{2}{3}hk_1, t_{n-1}+\frac{2}{3}h)\\
%v_n &= v_{n-1} + h(\frac{1}{4}k_1+\frac{3}{4}k_2)
%\end{align*}
%
%Converting back to original variables:
%\begin{align*}
%k_1 &= f(v_{n-1}, t_{n-1}) = y_{n-1}\\
%k_2 &= f(v_{n-1}+\frac{2}{3}hk_1, t_{n-1}+\frac{2}{3}h)\\
%&= e^{-\frac{2}{3}hL}N\left(e^{\frac{2}{3}hL}(y_{n-1}+\frac{2}{3}hk_1)\right)\\
%y_n &= e^{hL}v_n = e^{hL}\left(y_{n-1} + h(\frac{1}{4}k_1+\frac{3}{4}k_2)\right)
%\end{align*}

\paragraph{IF-RK4 (Classic)} This is the classic 4th order explicit Runge-Kutta method.
\begin{align*}
k_1 &= f(v_{n-1}, t_{n-1})\\
k_2 &= f(v_{n-1}+\frac{h}{2}k_1, t_{n-1}+\frac{h}{2})\\
k_3 &= f(v_{n-1} + \frac{h}{2}k_2, t_{n-1}+\frac{h}{2})\\
k_4 &= f(v_{n-1}+hk_3, t_{n-1}+h)\\
v_n &= v_{n-1} + h(\frac{1}{6}k_1+\frac{1}{3}k_2+\frac{1}{3}k_3+\frac{1}{6}k_4)
\end{align*}

Converting back to original variables:
\begin{align*}
k_1 &= f(v_{n-1}, t_{n-1})=N(y_{n-1})\\
k_2 &= f(v_{n-1}+\frac{h}{2}k_1, t_{n-1}+\frac{h}{2}) = e^{-\frac{1}{2}hL}N\left(e^{\frac{1}{2}hL}(y_{n-1}+\frac{h}{2}k_1)\right)\\
k_3 &= f(v_{n-1} + \frac{h}{2}k_2, t_{n-1}+\frac{h}{2})= e^{-\frac{1}{2}hL}N\left(e^{\frac{1}{2}hL}(y_{n-1}+\frac{h}{2}k_2)\right)\\
k_4 &= f(v_{n-1}+hk_3, t_{n-1}+h) = e^{-hL}N\left(e^{hL}(y_{n-1}+hk_3)\right)\\
y_n &= e^{hL}v_n =e^{hL} \left(y_{n-1} + h(\frac{1}{6}k_1+\frac{1}{3}k_2+\frac{1}{3}k_3+\frac{1}{6}k_4)\right)
\end{align*}


\section{Resonant Triad}
We study the Resonant Triad system ODEs given by Equations~\ref{eqn:RT1} to \ref{eqn:RT3}, which can be summarized in vector form, Equation~\ref{eqn:RT}.
\begin{align}
\ddt z_1 &= i\omega_1 z_1 + \epsilon C_1 z_2^*z_3^*\label{eqn:RT1} \\
\ddt z_2 &= i\omega_1 z_2 + \epsilon C_2 z_1^*z_3^*\label{eqn:RT2} \\
\ddt z_3 &= i\omega_1 z_3 + \epsilon C_3 z_1^*z_2^*\label{eqn:RT3} \\
\ddt \vec{z} &= Lz + N(z) \label{eqn:RT}
\end{align}

The linear operator for Equations~\ref{eqn:RT} is given below.
\begin{equation}
L = \begin{bmatrix}
i\omega_1 & 0 & 0\\
0 &i\omega_2 & 0 \\
0 & 0 & i\omega_3 
\end{bmatrix}
\end{equation}

Since $L$ is a diagonal matrix, its exponential is easy to calculate. 
\begin{equation}
e^{hL} = \begin{bmatrix}
e^{ih\omega_1} & 0 & 0\\
0 &e^{ih\omega_2} & 0 \\
0 & 0 & e^{ih\omega_3} 
\end{bmatrix}
\end{equation}

The nonlinear function is then, 
\begin{equation}
N(\vec{z}) = \epsilon \begin{bmatrix}
C_1 z_2^*z_3^* \\
C_2 z_1^*z_3^* \\
C_3 z_1^*z_2^* \\
\end{bmatrix}
\end{equation}
 
Let's apply the IF change of variables with $v(t) = e^{(t_{n-1}-t)L}z(t)$.
Then, the ODE becomes, 
\begin{equation}
\ddt \vec{v} = f(\vec{v},t_{n-1}+\tau) = e^{-\tau L} N(e^{\tau L} v).
\label{eqn:tnIFRT}
\end{equation}

It turns out that $f(\vec{v}, t_{n-1}+\tau)$ can actually be represented just as a function of $v$. 
Some algebra to show this:
\begin{align*}
f(\vec{v},t_{n-1}+\tau) &= e^{-\tau L} N(e^{\tau L} \vec{v}) \\
&= e^{-\tau L} N\left(\begin{bmatrix}
e^{i\tau\omega_1} & 0 & 0\\
0 &e^{i\tau\omega_2} & 0 \\
0 & 0 & e^{i\tau\omega_3} 
\end{bmatrix}\begin{bmatrix}
v_1 \\
v_2\\
v_3\\
\end{bmatrix}\right) =e^{-\tau L}  N\left(\begin{bmatrix}
e^{i\tau\omega_1}v_1 \\
e^{i\tau\omega_2}v_2\\
e^{i\tau\omega_3}v_3\\
\end{bmatrix}\right) \\
&= e^{-\tau L} \epsilon \begin{bmatrix}
C_1e^{-i\tau(\omega_2+\omega_3)}v_2^*v_3^* \\
C_2e^{-i\tau(\omega_1+\omega_3)}v_1^*v_3^* \\
C_1e^{-i\tau(\omega_1+\omega_2)}v_1^*v_2^* \\
\end{bmatrix} = e^{-\tau L} \epsilon \begin{bmatrix}
C_1e^{i\tau(\omega_1)}v_2^*v_3^* \\
C_2e^{i\tau(\omega_2)}v_1^*v_3^* \\
C_3e^{i\tau(\omega_3)}v_1^*v_2^* \\
\end{bmatrix} \text{, since }\omega_1+\omega_2+\omega_3 = 0,\\
&= e^{-\tau L} \begin{bmatrix}
e^{i\tau\omega_1} & 0 & 0\\
0 &e^{i\tau\omega_2} & 0 \\
0 & 0 & e^{i\tau\omega_3} 
\end{bmatrix} \epsilon\begin{bmatrix}
C_1v_2^*v_3^* \\
C_2v_1^*v_3^* \\
C_3v_1^*v_2^* \\
\end{bmatrix} = e^{-\tau L} e^{\tau L} N(\vec{v}) = N(\vec{v})
\end{align*}

So we actually end up with an autonomous system of ODEs in the new variable, $\vec{v}$ as well. 
Let's re-do the discretizations by using $\ddt{\vec{v}} = f(\vec{v},t)=N(\vec{v})$.

\paragraph{IF-Euler} Applying explicit Euler to Equation~\ref{eqn:tnIF} results in,
\begin{equation*}
v_n = v_{n-1} + h f(v_{n-1, t_{n-1}}) = v_{n-1} + h N(v_{n-1}). 
\end{equation*}

Returning to the original variables gives us the full IF-Euler discretization.
\begin{equation}
y_n = e^{hL}v_n =e^{hL} (y_{n-1}+hN(y_{n-1}))
\end{equation}

\paragraph{IF-RK2 (Heun/Trapezoidal)} This is one of the simplest non-trivial Runge-Kutta method. 
\begin{align*}
k_1 &= N(v_{n-1})\\
k_2 &= N(v_{n-1}+hk_1)\\
v_n &= v_{n-1} + h\left(\frac{1}{2}k_1 +\frac{1}{2}k_2\right)
\end{align*}

Converting back to original variables:
\begin{align*}
k_1 &= N(y_{n-1})\\
k_2 &= N(y_{n-1}+hk_1)\\
y_n &= e^{hL}v_n  = e^{hL}\left(y_{n-1} + h\left(\frac{1}{2}k_1 +\frac{1}{2}k_2\right)\right)\\
\end{align*}
%\paragraph{IF-RK2 (Explicit Midpoint Method)} This is one of the simplest non-trivial Runge-Kutta method. 
%\begin{align*}
%k_1 &= f(v_{n-1}, t_{n-1})\\
%k_2 &= f(v_{n-1}+\frac{h}{2}k_1, t_{n-1}+\frac{h}{2})\\
%v_n &= v_{n-1} + hk_2
%\end{align*}
%
%Converting back to original variables:
%\begin{align*}
%k_1 &= f(v_{n-1}, t_{n-1}) = N(y_{n-1})\\
%k_2 &= f(v_{n-1}+\frac{h}{2}k_1, t_{n-1}+\frac{h}{2})\\
%&= e^{-\frac{1}{2}hL}N(e^{\frac{1}{2}hL}(y_{n-1}+\frac{h}{2}k_1))\\
%y_n &= e^{hL}v_n = e^{hL}(y_{n-1} + hk_2)
%\end{align*}
%
%\paragraph{IF-RK2 (Ralston's Method)} This is one of the simplest non-trivial Runge-Kutta method. 
%\begin{align*}
%k_1 &= f(v_{n-1}, t_{n-1})\\
%k_2 &= f(v_{n-1}+\frac{2}{3}hk_1, t_{n-1}+\frac{2}{3}h)\\
%v_n &= v_{n-1} + h(\frac{1}{4}k_1+\frac{3}{4}k_2)
%\end{align*}
%
%Converting back to original variables:
%\begin{align*}
%k_1 &= f(v_{n-1}, t_{n-1}) = y_{n-1}\\
%k_2 &= f(v_{n-1}+\frac{2}{3}hk_1, t_{n-1}+\frac{2}{3}h)\\
%&= e^{-\frac{2}{3}hL}N\left(e^{\frac{2}{3}hL}(y_{n-1}+\frac{2}{3}hk_1)\right)\\
%y_n &= e^{hL}v_n = e^{hL}\left(y_{n-1} + h(\frac{1}{4}k_1+\frac{3}{4}k_2)\right)
%\end{align*}

\paragraph{IF-RK4 (Classic)} This is the classic 4th order explicit Runge-Kutta method.
\begin{align*}
k_1 &= N(v_{n-1})\\
k_2 &= N(v_{n-1}+\frac{h}{2}k_1)\\
k_3 &= N(v_{n-1} + \frac{h}{2}k_2)\\
k_4 &= N(v_{n-1}+hk_3)\\
v_n &= v_{n-1} + h(\frac{1}{6}k_1+\frac{1}{3}k_2+\frac{1}{3}k_3+\frac{1}{6}k_4)
\end{align*}

Converting back to original variables:
\begin{align*}
k_1 &= N(y_{n-1})\\
k_2 &= N(y_{n-1}+\frac{h}{2}k_1)\\
k_3 &= N(y_{n-1} + \frac{h}{2}k_2)\\
k_4 &= N(y_{n-1}+hk_3)\\
y_n &= e^{hL} v_n = e^{hL} \left(y_{n-1} + h(\frac{1}{6}k_1+\frac{1}{3}k_2+\frac{1}{3}k_3+\frac{1}{6}k_4)\right)
\end{align*}


\end{document}